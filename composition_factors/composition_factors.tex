Ввиду сложности общего случая, связанной с $p$-адическим разложением числа $k$, 
здесь исследуется только случай $p$-ограниченного веса.

\begin{lemma}
Если $ \lambda_2 > 0 $, то
$$ \chi (\nabla(\lambda)) = (x_1 x_2)^{\lambda_2 - 1} x_3^{\lambda_3} p_{t + pk} (x_1, x_2) [\,x_1 x_2 + x_1 x_3 + x_2 x_3 + x_3^2\,], $$
Если $ \lambda_2 = 0 $, то
$$ \chi (\nabla(\lambda)) = x_3^{\lambda_3} [\,p_{t + pk} (x_1, x_2) + p_{t + pk - 1} (x_1, x_2) x_3 \,]. $$
\end{lemma}
\begin{proof}
Если $ \lambda_2 > 0 $, то $ v_i, w_i, u_i, r_i, \quad i = 0, \ldots \lambda_1 - \lambda_2 $ являются полиномами и составляют базис $ \nabla(\lambda) $, 
откуда следует первое утверждение леммы.

Если $ \lambda_2 = 0 $, то $ \lambda = (t, 0 \,|\, 0) + p (0, 0 \,|\, \lambda_3 ') $. 
В этом случае согласно \cite{la_scala} $ \quad \nabla(\lambda) = \nabla(t, 0 \,|\, 0) \otimes \nabla(\lambda_3 ')^p $.
Тогда $ \chi (\nabla(\lambda_3 ')^p) = x_3^{p\lambda_3 '} = x_3^{\lambda_3} $.

Базис $ \nabla(t, 0 \,|\, 0) $ получается из вариантов заполнения строки длины $t$ 
невозрастающей последовательностью индексов $ 1, 2, 3 $, причем повторяться могут только чётные, т.е. $1$ и $2$.
Каждый вариант заполнения соответсвует базисному моному по правилу 
$$ (\underbrace{1, \ldots, 1}_i, \underbrace{2, \ldots, 2}_{t - 1 - i}, 3) \longleftrightarrow c_{11}^i c_{12}^{t - 1 - i} c_{13}. $$
Подробное описание можно найти в \cite{la_scala}. 

Таким образом базис составляют мономы $ c_{11}^i c_{12}^(t - 1 - i) c_{13}, \quad i = 0, \ldots, t - 1 - i $ 
и мономы $ c_{11}^i c_{12}^{t - i}, \quad i = 0, \ldots, t $, откуда и получаем формальный характер.
\end{proof}

В случае существования фактора $ V/L_{\lambda} \cong L_{\mu} $ старший вектор $ L_{\mu} $ 
зануляется суперпроизводными $ {_{21}D}, {_{31}D}, {_{32}D} $ по модулю $ L_{\lambda} $. 
И обратно, если вектор из $ L_{\mu} $ зануляется суперпроизводными $ {_{21}D}, {_{31}D}, {_{32}D} $ по модулю $ L_{\lambda} $, 
то он является старшим вектором $ L_{\mu} $. При этом $ \chi (\mu) = \chi (\nabla(\lambda)) - \chi (\lambda) $. 
Таким образом, фактор $ L_{\mu} $ можно найти, зная формальные характеры $ \nabla(\lambda) $ и $ L_{\lambda} $. 

\begin{proposition}
Пусть $ \lambda = (\lambda_1, \lambda_2 \,|\, \lambda_3) $ -- $p$-ограниченный вес,  
$ V = \nabla(\lambda) $ -- соответствующий костандартный модуль, 
$ L_{\lambda} $ -- неприводимый модуль со старшим весом $\lambda$.

(a) Если $\lambda$ регулярный или $ \lambda_2 = 0 $, то $ V = L_{\lambda} $.

(b) Если $\lambda$ критический, то $ V/L_{\lambda} \simeq L_{\bar{\lambda}} $, 
где $ \bar{\lambda} = (\lambda_1 - 1, \lambda_2 \,|\, \lambda_3 + 1) $ -- критический, при этом $\mbox{dim}\, L_{\bar{\lambda}} = 2t + 1 $. 

(c) Если $\lambda$ сильно критический, то $ V/L_{\lambda} \simeq L_{\hat{\lambda}} $, 
где $ \hat{\lambda} = (\lambda_1, \lambda_2 - 1 \,|\, \lambda_3 + 1) $ -- сильно критический, при этом $\mbox{dim}\, L_{\hat{\lambda}} = 2t + 3 $.
\end{proposition}
\begin{proof} Так как $\lambda$ -- $p$-ограниченный вес, то $ p_{t + pk} (x_1, x_2) = p_{t} (x_1, x_2) $ и \\
$ \prod\limits_{i = 0}^{s}p_{k_i} ~(x_1^{p^{i + 1}}, x_2^{p^{i + 1}}) = 1 $. 
Поэтому в случае регулярного веса $\lambda$ или $ \lambda_2 = 0 $ формальные характеры $V$ и $ L_{\lambda} $ совпадают, 
следовательно совпадают и сами модули.

(b) $\lambda$ -- критический. Напомним, что в этом случае базис $ L_{\lambda} $ составляют векторы $ v_0, \ldots, v_t, u_0, w_t, q_0, \ldots, q_{t - 1} $.
\\
$ \chi(V) - \chi(L_{\lambda}) = (x_1 x_2)^{\lambda_2 - 1} x_3^{\lambda_3} p_t (x_1, x_2) [\,x_1 x_2 + x_1 x_3 + x_2 x_3 + x_3^2\,] - \\ 
(x_1 x_2)^{\lambda_2 - 1} x_3^{\lambda_3} [\,p_t (x_1, x_2) x_1 x_2 + p_{t + 1} (x_1, x_2) x_3\,] = (x_1 x_2)^{\lambda_2 - 1} x_3^{\lambda_3} 
[\,\sum \limits_{i = 0}^{t - 1} x_1^{t - j} x_2^{j + 1} x_3 + \sum \limits_{i = 0}^t x_1^{t - j} x_2^j x_2^2 \,] $.
Отсюда следует, что старший вектор фактора $ U_0 \cong V/L_{\lambda} $ имеет вес 
$ \bar{\lambda} = (t + \lambda_2 - 1, \lambda_2 \,|\, \lambda_3 + 1) = (\lambda_1 - 1, \lambda_2 \,|\, \lambda_3 + 1) $, который является критическим.
Вектор $ w_0 $ веса $ \bar{\lambda} $ зануляется соответствующими производными (см. список значений производных), 
т.е. $ w_0 $ -- старший вектор $ U_0 $. $ w_0 $ под действием производных $ {_{12}D}, {_{13}D}, {_{23}D} $ 
порождает векторы $ w_0, \ldots, w_{t - 1}, r_0, \ldots, r_t $ (по модулю $ L_{\lambda} $). 
Сравнивая размерности $ U_0 $ и $L_{\bar{\lambda}} $, получаемш $ U_0 = L_{\bar{\lambda}} $.

(c) $\lambda$ -- сильно критический. Базис $ L_{\lambda} $ составляют векторы $ v_0, \ldots, v_t, q_0, \ldots, q_{t - 1} $.
\\
$ \chi(V) - \chi(L_{\lambda}) = (x_1 x_2)^{\lambda_2 - 1} x_3^{\lambda_3} p_t (x_1, x_2) [\,x_1 x_2 + x_1 x_3 + x_2 x_3 + x_3^2\,] - \\
(x_1 x_2)^{\lambda_2} x_3^{\lambda_3} [\,p_t (x_1, x_2) + p_{t - 1} (x_1, x_2) x_3 \,] =  (x_1 x_2)^{\lambda_2 - 1} x_3^{\lambda_3} 
[\, \sum \limits_{i = 0}^t x_1^{t - j} x_2^{j + 1} x_3 + x_1^{t + 1} x_3 + \sum \limits_{i = 0}^t x_1^{t - j} x_2^j x_3^2 \,] $.
Старший вектор фактора $ U_1 \cong V/L_{\lambda} $ имеет вес 
$ \hat{\lambda} = (t + \lambda_2, \lambda_2 - 1 \,|\, \lambda_3 + 1) = (\lambda_1, \lambda_2 - 1 \,|\, \lambda_3 + 1) $, 
который является сильно критическим. Вектор $ u_0 $ веса $ \hat{\lambda} $ зануляется производными, 
следовательно, $ u_0 $ -- старший вектор $ U_1 $. Аналогично, $ U_1 = L_{\hat{\lambda}} $.
\end{proof}
