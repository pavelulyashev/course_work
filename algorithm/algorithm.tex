Обозначим $ \chi(\xi_{\lambda}) = \chi (A(r)\xi_{\lambda}), \quad \chi(I_{\lambda}) = \chi (I(\lambda)) $.

Для самого старшего вектора в иерархии $ \lambda' = (r, 0 \,|\, 0) $ только $ d_{\lambda', \lambda'} = 1 $, поэтому
$ \chi(\xi_{\lambda'}) = \chi(I_{\lambda'}) $. Для его последователя $ \lambda'' $ верно 
$ \chi(\xi_{\lambda''}) = \chi(I_{\lambda''}) + d_{\lambda', \lambda''} \chi(I_{\lambda'}) $, откуда имеем
$ \chi(I_{\lambda''}) = \chi(\xi_{\lambda''}) - d_{\lambda', \lambda''} \chi(I_{\lambda'}) $.

Итерируя эту процедуру, можно вычислить $ \chi(\xi_{\mu}) $ для любого $\mu \in \Lambda(r)^+ $.
Таким образом, имеем алгоритм вычисления $ \chi(\xi_{\mu}) $:

1) Строим иерархию весов из $ \Lambda(r)^+ $.
\\
Пример иерархии для $ r = 9, p = 11 $.
$$
\begin{matrix}
(9, 0 | 0) &            &            &            &            &            &            &            \\
(8, 1 | 0) &            &            &            &            &            &            &            \\
(7, 2 | 0) &            &            &            &            &            &            &            \\
(6, 3 | 0) & (7, 1 | 1) &            &            &            &            &            &            \\
(5, 4 | 0) & (6, 2 | 1) &            &            &            &            &            &            \\   
           & (5, 3 | 1) & (6, 1 | 2) &            &            &            &            &            \\  
           & (4, 4 | 1) & (5, 2 | 2) &            &            &            &            &            \\  
           &            & (4, 3 | 2) & (5, 1 | 3) &            &            &            &            \\ 
           &            &            & (4, 2 | 3) &            &            &            &            \\ 
           &            &            & (3, 3 | 3) & (4, 1 | 4) &            &            &            \\
           &            &            &            & (3, 2 | 4) &            &            &            \\ 
           &            &            &            &            & (3, 1 | 5) &            &            \\
           &            &            &            &            & (2, 2 | 5) &            &            \\
           &            &            &            &            &            & (2, 1 | 6) &            \\ 
           &            &            &            &            &            &            & (1, 1 | 7) \\ 
\end{matrix}
$$
Последователи расположены друг за другом по вертикали и вправо-вниз по диагонали. 
Веса, находящиеся в одном горизонтальном ряду, не сравнимы.

2) Для вычисления $ \chi(\xi_{\lambda}) $ нужно посчитать $ \chi(\xi_{\mu}) $ для всех предшественников $\mu$, начиная от вершины.

3) Выразить $ \chi(\xi_{\lambda}) $ через $ \chi(\xi_{\mu}) $ для предшественников $\mu$ из двух предшествующих столбцов. 
При этом если $\lambda$ и $\mu$ находятся в одном столбце или через столбец, то $ d_{\mu, \lambda} = 1 $. 
Если $\lambda$ находится вправо-вниз по диагонали от $\mu$, то $ d_{\mu, \lambda} = 1 $. 
Для всех остальных предшественников $\mu$ в соседнем столбце $ d_{\mu, \lambda} = 2 $.
