\begin{proposition}
Для произвольного веса $ \lambda \in \Lambda(r) $ справедливо разложение \\
$ A(r)\xi_{\lambda} = \bigoplus_{\mu in \Lambda(r)^+} I(\mu)^{d_{\mu, \lambda}} $, 
где $ d_{\mu, \lambda} = \mbox{\textup{dim}}\, L(\mu)_{\lambda} $.
\end{proposition}

$ A(r)\xi_{\lambda} $ -- подпространство в $ A(r) $, образованное всеми мономами, имеющими вес слева $ \lambda $. 
Подробнее теоретический материал можно найти в \cite{la_scala}, \cite{borel_subalgebras}.

Базисные элементы известны, поэтому можно записать формальный характер $ A(r)\xi_{\lambda} $.
Мы описали формальные характеры неприводимых модулей $ L(\mu) $, поэтому можем вычислить $ d_{\mu, \lambda} $ для произвольного веса $ \lambda $.
Таким образом, при определенных условиях можно вычислить формальные характеры инъективных модулей $ I(\mu) $.

Пусть $ r < p $. Тогда все веса из $ \Lambda(r)^+ $ будут $p$-ограниченными, поэтому все размерности $ d_{\mu, \lambda} $ можно легко вычислить.
