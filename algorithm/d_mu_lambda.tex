Напомним, что вес $ \lambda = (\lambda_1, \lambda_2 \,|\, \lambda_3) $ является старшим, 
если $ \lambda_1 \geqslant \lambda_2 $ и $ p \,|\, \lambda_3 $, если $ \lambda_2 = 0 $.

\begin{lemma}
При $ r < p $ любой старший вес $ \lambda = (\lambda_1, \lambda_2 \,|\, \lambda_3) $ степени $r$ 
является регулярным за исключением $ (r, 0 \,|\, 0) $.
\end{lemma}
\begin{proof}
Если $\lambda$ -- сильно критический, то $ \lambda_2 + \lambda_3 = 0 $, т.к. $ \lambda_2 + \lambda_3 < p $. 
Тогда $ \lambda_2 = \lambda_3 = 0 $, т.е. $ \lambda = (r, 0 \,|\, 0) $.

Если $\lambda$ -- критический, то $ \lambda_1 + \lambda_3 + 1 = p $, т.е. $ \lambda_1 + \lambda_3 = p - 1 $,
но тогда $ \lambda_2 = 0 $, $ \lambda_3 = 0 $ и вес не является критическим.
\end{proof}

Для определения ненулевых размерностей $ d_{\mu, \lambda} $ составим иерархию всех весов из $ \Lambda(r)^+ $.

\begin{definition}
Назовем вес $ \mu $ последователем веса $ \lambda $, если $ d_{\lambda, \mu} > 0 $ и $ \nexists \eta \in \Lambda(r)^+ : \lambda > \eta > \mu $.
\end{definition}
Для веса $ (r, 0 \,|\, 0) $ последователем является только $ (r - 1, 1 \,|\, 0) $ 

\begin{lemma}
Пусть $ \lambda \in \Lambda(r)^+ $ -- регулярный вес. Тогда его последователями являются несравнимые веса
$ (\lambda_1 - 1, \lambda_2 + 1 \,|\, \lambda_3) $ и $ (\lambda_1, \lambda_2 - 1 \,|\, \lambda_3 + 1) $.
\end{lemma}
\begin{proof}
$ L_{\lambda} $ имеет базис $ v_i, w_i, u_i, r_i $ с весами 
$ (\lambda_1 - i, \lambda_2 + i \,|\, \lambda_3), (\lambda_1 - i - 1, \lambda_2 + i \,|\, \lambda_3 + 1), 
(\lambda_1 - i, \lambda_2 + i - 1 \,|\, \lambda_3 + 1), (\lambda_1 - i - 1, \lambda_2 + i - 1 \,|\, \lambda_3 + 2) $.
Подразумевая сравнение весов, имеем $ v_1 > v_i, u_0 > u_i, w_0 > w_i, r_0 > r_i $, поэтому последователями $ \lambda $ 
могут быть только веса векторов $ v_1, w_0, u_0, r_0 $. 
Поскольку $ v_1 > w_0, v_1 > r_0, u_0 > w_0, u_0 > r_0 $, а $ v_1 $ и $ u_0 $ не сравнимы, то 
последователями $ \lambda $ являются веса векторов $ v_1 $ и $ u_0 $ -- 
$ (\lambda_1 - 1, \lambda_2 + 1 \,|\, \lambda_3) $ и $ (\lambda_1, \lambda_2 - 1 \,|\, \lambda_3 + 1) $.
\end{proof}
%
\begin{remark}
Если $ \lambda = (\lambda_1, 1 \,|\, \lambda_3) $, то последователь будет только один -- \\
$ (\lambda_1 - 1, \lambda_2 + 1 \,|\, \lambda_3) $.
Если $ \lambda_1 = \lambda_2 $, то последователем будет только $ (\lambda_1, \lambda_2 - 1 \,|\, \lambda_3 + 1) $.
Таким образом, иерархия заканчивается, когда оба условия выполнены, т.е. последним весом будет $ (1, 1 \,|\, r - 2) $.
\end{remark}
%
\begin{lemma}
Пусть $ \lambda = (\lambda_1, \lambda_2 \,|\, \lambda_3), \mu = (\mu_1, \mu_2 \,|\, \mu_3) \in \Lambda(r)^+ $.

a) $ d_{\lambda, \mu} = 1 $, если выполнено одно из условий: \\
1) $ \mu_3 = \lambda_3 $; \\
2) $ \mu_3 = \lambda_3 + 2 $; \\
3) $ \mu_3 = \lambda_3 + 1, \mu_1 = \lambda_1 $; \\
4) $ \mu_3 = \lambda_3 + 1, \mu_2 = \lambda_1 $. \\

b) $ d_{\lambda, \mu} = 1 $, если $ \mu_3 = \lambda_3 + 1, \mu_1 \neq \lambda_1, \mu_2 \neq \lambda_1 $.

c) Иначе $ d_{\lambda, \mu} = 0 $.
\end{lemma}
\begin{proof}
В $ L_{\lambda} $ повторяются только веса векторов $ u_i $ и $ w_{i + 1} $.
Для векторов $ v_i $ $ \mu_3 = \lambda_3 $, для векторов $ r_i $ $ \mu_3 = \lambda_3 + 2 $. 
Также не повторяются веса векторов $ w_0, u_t $ -- $ (\lambda_1, \lambda_2 - 1 \,|\, \lambda_3 + 1), (\lambda_2 - 1, \lambda_1 \,|\, \lambda_3 + 1) $,
т.е. $ \mu_3 = \lambda_3 + 1 $ и $ \mu_1 = \lambda_1 $ или $ \mu_2 = \lambda_1 $.

Все остальные веса с $ \mu_3 = \lambda_3 + 1 $ имеют кратность 2.
\end{proof}
