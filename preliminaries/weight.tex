
\begin{definition}
Кортеж длины $r \quad I = (i_1, \ldots, i_r) $, где $ i_k \in \{1, 2, \ldots, m+n\} $, называется мультииндексом длины $r$. 
Множество всех мультииндексов обозначим $ I(m \,|\, n, r) $.
Весом мультииндекса $I$ называется кортеж неотрицательных целых чисел $ \lambda (I) = (\lambda_1, \ldots, \lambda_{m + n}) $, 
где $ \lambda_j = | \{k \;|\; i_k = j\} |$. Очевидно, что $\sum \lambda_i = r $. 
\end{definition}
Множество всех весов обозначим $ \Lambda(r) = \Lambda(m\,|\,n, r) $. 
Ясно, что $ \lambda(I) = \lambda(J) \iff I = ~J\pi $ для подходящей перестановки $ \pi \in S_{m+n} $. \\
Для любых мультииндексов $ I, J  \in I(m \,|\, n, r) $ определим линейный функционал $\xi_{IJ}$ в $ S(r) = A(r)^* $, дуальный моному $ c_{IJ} $.
Поскольку $ \xi_{II} = \xi_{I_{\pi} I_{\pi}} $, где $ \pi \in S_{m + n} $, этот элемент однозначно определяется 
весом $ \lambda = \lambda(I) $ и обозначается $ \xi_{\lambda} $.
На множестве весов $\Lambda(r)$ определим \textit{доминантный} порядок по правилу $ \mu \leq \lambda $, 
если $ \sum\limits_{1 \leqslant k \leqslant l} \mu_k \leq \sum\limits_{1 \leqslant k \leqslant l} \lambda_k, \quad 1 \leqslant l \leqslant m + n $.
\begin{theorem}
Если $V$ -- простой $S(r)$-модуль, то найдется вес $\lambda \in \Lambda(r)$ такой, что $ V_{\lambda} $ -- простой цоколь $ B(r)^+ $-модуля $V$. 
Для произвольного другого веса $ \mu \neq \lambda $ из $ V_{\mu} $ следует, что $ \mu < \lambda $ относительно доминантного порядка.
\end{theorem}
Вес, фигурирующий в теореме, и любой ненулевой вектор из $ V_{\lambda} $ называются \textit{старшим весом} и \textit{вектором} простого модуля $V$.
Вес называется \textit{допустимым}, если существует простой $ S(r) $-модуль со старшим весом $\lambda$. 
Простые $S(r)$-модули определяются своими  старшими весами однозначно. Поэтому простой модуль со старшим весом $\lambda$ обозначим $L(\lambda)$. 
Подножество в $\Lambda(r)$, состоящее из всех допустимых весов, обозначим $ \Lambda(r)^+ = \Lambda(m\,|\,n, r)^+ $.
