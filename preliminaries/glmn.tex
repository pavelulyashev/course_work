Взято из \cite{some_properties_supergroups}.

Обозначим через $ E(m\,|\,n) $ суперпространство с базисом $ e_1, \ldots, e_m, e_{m + 1}, \ldots, e_{m + n} $ 
с чётной частью $ e_1, \ldots, e_m $ и нечётной частью $ e_{m + 1}, \ldots, e_{m + n} $. 
Можно считать, что натуральным числам от 1 до $m + n$ приписана чётность по тому же правилу, то есть от 0 до $m$ -- чётные и от $m + 1$ до $m + n$ -- нечётные.
Определим супералгебру $ A(m\,|\,n) $ при помощи порождающих $x_{ij}$ и определяющих соотношений $x_{ij} x_{kl} - (-1)^{|x_{ij} x_{kl}|} x_{kl} x_{ij} = 0 $, 
где $ |x_{ij} \equiv |i| + |j| ~(\mbox{mod}\,2), \quad 1 \leqslant i, j, k, l \leqslant m + n $. 
Алгебра $A$ наделяется структурой супербиалгебры относительно коумножения, определенного на порождающих по правилу 
$ \delta_A (x_{ij}) = \sum\limits_{1 \leqslant k \leqslant m + n} x_{ik} \otimes x_{kj} $. 
Коединица задается как $ \epsilon_A (x_{ij}) = \delta_{ij}, \quad 1 \leqslant i, j \leqslant m + n $. 

Произвольная однородная компонента $ A(r) = A(m\,|\,n, r) $ является конечномерной суперкоалгеброй, а дуальное пространство $ A(r)^* $ -- супералгеброй, 
которая называется \textit{супералгеброй Шура} и обозначается $ S(m\,|\,n, r) $.

Матрицу из порождающих $x_{ij}$ обозначим $X$. Её блоки размера $ m \times m, m \times n, n \times m, n \times n $ обозначим соответственно 
$ X_{11}, X_{12}, X_{21}, X_{22} $. Таким образом,
$$ X = 
\begin{pmatrix}
X_{11} & X_{12} \\
X_{21} & X_{22} \\
\end{pmatrix}.
$$

Локализуя $ A(m\,|\,n) $ по чётному элементу $ d = d_1 d_2, \: d_1 = det(X_{11}), \: d_2 = det(X_{22}) $, получим супералгебру Хопфа, 
которая по определению является координатной алгеброй общей линейной супергруппы $ GL(m\,|\,n) $, т.е. $ A(m\,|\,n)_d = K[GL(m\,|\,n)] $.

Gl(m\,|\,n) не является группой в обычном смысле, а является функтором, сопоставляющим произвольной коммутативной супералгебре $A$
группу $ GL(m\,|\,n)(A) $, состоящую из всех обратимых $ (m + n) \times (m + n) $ матриц вида
$$ A = 
\begin{pmatrix}
A_{11} & A_{12} \\
A_{21} & A_{22} \\
\end{pmatrix},
$$
где $ A_{11} \in GL(m)(A_0), \: A_{22} \in GL(n)(A_0)$, а коэффициенты блоков $ A_{12}, A_{21} $ нечётны. 

