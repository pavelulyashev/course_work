Взято из \cite{donkin}.

Приведем определения супераналогов некоторых алгебраических систем. В общем случае суперизация достигается 
засчет введения $\mathbb{Z}_2$-градуировки, относительно которой все структурные функции однородны.

\begin{definition}
Суперпространством над полем $K$ называется векторное пространство $V$ с разложением $ V = V_0 \oplus V_1 $. $V_0$ называется чётной частью $V$, $V_1$ -- нечётной. 
Говорят, что элемент $v \in V$ является чётным, если $v \in V_0$, и нечетным, если $v \in V_1$.
\end{definition}
\begin{definition}
Подсуперпространством $V$ называется подпространство $U$ такое, что $ U = (U \cap V_0) \oplus (U \cap V_1) $.
\end{definition}
Если $U$ -- суперподпространство в $V$, то $U$ и $V/U$ являются суперпространствами.
\begin{definition}
Супералгеброй называется ассоциативная алгебра $A$ со структурой суперпространства $ A = A_0 \oplus A_1 $, 
при этом $ \forall\, a,b \in A \quad |ab| = |a| + |b| ~(\mbox{\textit{mod}}\,2) $.
Супералгебра называется коммутативной, если $ \forall\, a,b \in A \quad ab = (-1)^{|a| |b|} ba $.
\end{definition}
Здесь прямыми скобками обозначена четность соответствующего элемента.
Суперидеал $A$ -- это идеал алгебры, который одновременно является ее суперподпространством. 
Суперподалгебра -- подалгебра, одновременно являющаяся суперподпространством. 
\begin{definition}
Пусть $A$ -- супералгебра. Супермодулем называется $A$-модуль $V$, которые также является суперпространством, 
причем $ \forall \; a \in A \quad \forall v \in V \quad |av| = |a| ~+ ~|v| ~(\mbox{\textup{mod}}\,2) $.
\end{definition}

