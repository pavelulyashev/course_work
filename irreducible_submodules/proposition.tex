\begin{proposition} Пусть $ k = \sum\limits_{i = 0}^{s} k_i p^i $. Обозначим $ t_k = (k_0 + 1) \ldots (k_s + 1) $.
\begin{enumerate}
\item [\textup{(a)}] Если $\lambda$ - регулярный вес, то 

$$ \chi(\lambda) = (x_1 x_2)^{\lambda_2 - 1} x_3^{\lambda_3} p_t (x_1, x_2) [\,x_1 x_2 + x_1 x_3 + x_2 x_3 + x_3^2\,] 
\prod\limits_{i = 0}^{s}p_{k_i} ~(x_1^{p^{i + 1}}, x_2^{p^{i + 1}})
$$

и $ \mbox{\textup{dim}}\,(L_{\lambda}) = 4 (t + 1) t_k $.

\item [\textup{(b)}] Если $\lambda$ - критический вес, то

$$ \chi(\lambda) = (x_1 x_2)^{\lambda_2 - 1} x_3^{\lambda_3} [\,p_t (x_1, x_2) x_1 x_2 + p_{t + 1} (x_1, x_2) x_3\,] 
\prod\limits_{i = 0}^{s}p_{k_i} ~(x_1^{p^{i + 1}}, x_2^{p^{i + 1}}) 
$$

и $\mbox{\textup{dim}}\,(L_{\lambda}) = (2 t + 3) t_k $.
\item [\textup{(c)}] Если $\lambda$ - критический вес, то

$$ \chi(\lambda) = (x_1 x_2)^{\lambda_2} x_3^{\lambda_3} [\,p_t (x_1, x_2) + p_{t - 1} (x_1, x_2) x_3 \,] 
\prod\limits_{i = 0}^{s}p_{k_i} ~(x_1^{p^{i + 1}}, x_2^{p^{i + 1}})
$$

и $\mbox{\textup{dim}}\,(L_{\lambda}) = (2 t + 1) t_k $.
\end{enumerate}
\end{proposition}

Для доказательства достаточно доказать утверждение для $p$-ограниченной части.
\begin{lemma} 
\begin{enumerate} Пусть $ \lambda = (t + \lambda_2, \lambda_2 \,|\, \lambda_3), \quad 0 \leq t < p $.
\item [\textup{(a)}] Если $\lambda$ - регулярный вес, то 

$$ \chi(\lambda) = (x_1 x_2)^{\lambda_2 - 1} x_3^{\lambda_3} p_t (x_1, x_2) [\,x_1 x_2 + x_1 x_3 + x_2 x_3 + x_3^2\,] $$

и $ \mbox{\textup{dim}}\,(L_{\lambda}) = 4 (t + 1) t_k $.

\item [\textup{(b)}] Если $\lambda$ - критический вес, то

$$ \chi(\lambda) = (x_1 x_2)^{\lambda_2 - 1} x_3^{\lambda_3} [\,p_t (x_1, x_2) x_1 x_2 + p_{t + 1} (x_1, x_2) x_3\,] $$

и $\mbox{\textup{dim}}\,(L_{\lambda}) = (2 t + 3) t_k $.
\item [\textup{(c)}] Если $\lambda$ - критический вес, то

$$ \chi(\lambda) = (x_1 x_2)^{\lambda_2} x_3^{\lambda_3} [\,p_t (x_1, x_2) + p_{t - 1} (x_1, x_2) x_3 \,] $$

и $\mbox{\textup{dim}}\,(L_{\lambda}) = (2 t + 1) t_k $.
\end{enumerate}
\end{lemma}

\begin{proof}

$\lambda_2 > 0$. Тогда векторы $v_i, w_i, u_i \mbox{ и } r_i$ полиномиальны для $i = 0, \ldots, \lambda_1 - \lambda_2$ и образуют базис модуля $\nabla (\lambda)$.
Базис $ L_{\lambda} $ составляют векторы, порожденные из старшего вектора суперпроизводными $ _{12}D, _{13}D, _{23}D $.

Вычислим $v_i^{_{13}D}$. Запишем вспомогательные равенства, которые понадобятся далее:
\\
$ d y_1 y_2 = \frac{ (c_{22}c_{13} - c_{12}c_{23}) (c_{11}c_{23} - c_{21}c_{13}) }{d} = \frac{ c_{22}c_{13}c_{11}c_{23} + c_{12}c_{23}c_{21}c_{13} }{d} = c_{13}c_{23}, $
\\
$ d c_{31} y_1 c_{32} y_2 = - c_{31}c_{32} d y_1 y_2 = - c_{31}c_{32}c_{13}c_{23} = c_{31}c_{13}c_{32}c_{23}, $
\\
$ c_{13} (c_{31} y_1 + c_{32} y_2) = \frac{c_{13}c_{31} (c_{22}c_{13} - c_{12}c_{23}) + c_{13}c_{32} (c_{11}c_{23} - c_{21}c_{13}) }{d} = \frac{-c_{13}c_{23} (c_{32}c_{11} - c_{31}c_{12})}{d} = c_{11} c_{32} y_2 y_1 + c_{12}c_{31} y_1 y_2, $
\\
$ c_{11} y_1 + c_{12} y_2 = \frac{c_{11} (c_{22}c_{13} - c_{12}c_{23}) + c_{12} (c_{11}c_{23} - c_{21}c_{13}) }{d} = 
\frac{ c_{11}c_{22}c_{13} - c_{12}c_{21}c_{13} }{d} = c_{13}. $

Учитывая их, перепишем вектор $v_i$ в виде
\\
$ v_i = d^{\lambda_2 - 1} c_{11}^{\lambda_1 - \lambda_2 - i} c_{12}^i (c_{33}^{\lambda_3} d - \lambda_3 c_{33}^{\lambda_3 - 1} (c_{31} d y_1 + c_{32} d y_2) + \lambda_3 (\lambda_3 - 1) c_{33}^{\lambda_3 - 2} c_{31}c_{13}c_{32}c_{23}). $
\\
\\
%
% Vi 13 D
%
$ v_i^{_{13}D} = d^{\lambda_2 - 1} c_{11}^{\lambda_1 - \lambda_2 - i} c_{12}^i (c_{33}^{\lambda_3} y_1 d - \lambda_3 c_{33}^{\lambda_3 - 1} (- c_{33} d y_1 - 2 c_{32} c_{13} c_{23}) - \lambda_3 (\lambda_3 - 1) c_{33}^{\lambda_3 - 2} c_{33}c_{13}c_{32}c_{23}) + 
\\
(\lambda_1 - \lambda_2 - i) d^{\lambda_2 - 1} c_{11}^{\lambda_1 - \lambda_2 - i - 1} c_{13} c_{12}^i (c_{33}^{\lambda_3} d - \lambda_3 c_{33}^{\lambda_3 - 1} (c_{31} d y_1 + c_{32} d y_2) + \lambda_3 (\lambda_3 - 1) c_{33}^{\lambda_3 - 2} c_{31}c_{13}c_{32}c_{23}) +
\\
(\lambda_2 - 1) d^{\lambda_2 - 1} y_1 c_{11}^{\lambda_1 - \lambda_2 - i} c_{12}^i (c_{33}^{\lambda_3} d - \lambda_3 c_{33}^{\lambda_3 - 1} (c_{31} d y_1 + c_{32} d y_2) + \lambda_3 (\lambda_3 - 1) c_{33}^{\lambda_3 - 2} c_{31}c_{13}c_{32}c_{23}) = 
\\
t_1 + t_2 + t_3 = (*) $
\\
\\
$ t_1 = d^{\lambda_2 - 1} c_{11}^{\lambda_1 - \lambda_2 - i} c_{12}^i (c_{33}^{\lambda_3} y_1 d + \lambda_3 c_{33}^{\lambda_3} d y_1 + 2 \lambda_3 c_{33}^{\lambda_3 - 1} c_{32}c_{13}c_{23} + \lambda_3 (\lambda_3 - 1) c_{33}^{\lambda_3 - 1} c_{32}c_{13}c_{23}) =
d^{\lambda_2 - 1} c_{11}^{\lambda_1 - \lambda_2 - i} c_{12}^i ( (\lambda_3 + 1) c_{33}^{\lambda_3} y_1 d - (\lambda_3 + 1) \lambda_3 c_{33}^{\lambda_3 - 1} c_{32}c_{23}c_{13}) = (\lambda_3 + 1) w_i $
\\
\\
$ t_3 = (\lambda_2 - 1) d^{\lambda_2 - 1} y_1 c_{11}^{\lambda_1 - \lambda_2 - i} c_{12}^i (c_{33}^{\lambda_3} d - \lambda_3 c_{33}^{\lambda_3 - 1} c_{32} d y_2) = (\lambda_2 - 1) w_i $
\\
\\
$ t_2 = (\lambda_1 - \lambda_2 - i) d^{\lambda_2 - 1} c_{11}^{\lambda_1 - \lambda_2 - i - 1} c_{12}^i (c_{33}^{\lambda_3} c_{13} - \lambda_3 c_{33}^{\lambda_3 - 1} c_{13} (c_{31} y_1 + c_{32} y_2)) = 
\\
(\lambda_1 - \lambda_2 - i) d^{\lambda_2 - 1} c_{11}^{\lambda_1 - \lambda_2 - i - 1} c_{12}^i c_{33}^{\lambda_3} c_{13} + (\lambda_1 - \lambda_2 - i) d^{\lambda_2 - 1} c_{11}^{\lambda_1 - \lambda_2 - i} c_{12}^i (- \lambda_3 c_{33}^{\lambda_3 - 1} c_{32} y_2 y_1) + 
\\
(\lambda_1 - \lambda_2 - i) d^{\lambda_2 - 1} c_{11}^{\lambda_1 - \lambda_2 - i - 1} c_{12}^{i + 1} (- \lambda_3 c_{33}^{\lambda_3 - 1} c_{32} y_1 y_2) = 
(\lambda_1 - \lambda_2 - i) d^{\lambda_2 - 1} c_{11}^{\lambda_1 - \lambda_2 - i - 1} c_{12}^i c_{33}^{\lambda_3} c_{13} + 
\\
(\lambda_1 - \lambda_2 - i) w_i - (\lambda_1 - \lambda_2 - i) d^{\lambda_2 - 1} c_{11}^{\lambda_1 - \lambda_2 - i} c_{12}^i c_{33}^{\lambda_3} y_1 + (\lambda_1 - \lambda_2 - i) u_{i + 1} - 
\\
(\lambda_1 - \lambda_2 - i) d^{\lambda_2 - 1} c_{11}^{\lambda_1 - \lambda_2 - i - 1} c_{12}^{i + 1} c_{33}^{\lambda_3} y_2 = 
(\lambda_1 - \lambda_2 - i) w_i + (\lambda_1 - \lambda_2 - i) u_{i + 1} + 
\\
(\lambda_1 - \lambda_2 - i) d^{\lambda_2 - 1} c_{11}^{\lambda_1 - \lambda_2 - i - 1} c_{12}^i c_{33}^{\lambda_3} (c_{13} - c_{11} y_1 - c_{12} y_2) = (\lambda_1 - \lambda_2 - i) w_i + (\lambda_1 - \lambda_2 - i) u_{i + 1} $
\\
\\
$ (*) = (\lambda_3 + 1) w_i + (\lambda_2 - 1) w_i + (\lambda_1 - \lambda_2 - i) w_i + (\lambda_1 - \lambda_2 - i) u_{i + 1} = 
(\lambda_1 + \lambda_3 - i) w_i + (\lambda_1 - \lambda_2 - i) u_{i + 1}. $

\\

Аналогично вычисляются остальные производные.

$ v_i^{_{12}D} = (\lambda_1 - \lambda_2 - i)v_{i + 1}, $

$ v_i^{_{13}D} = (\lambda_1 + \lambda_3 - i)w_i + (\lambda_1 - \lambda_2 - i)u_{i + 1}, $

$ v_i^{_{23}D} = i w_{i - 1} + (\lambda_2 + \lambda_3 + i)u_i, $

$ v_i^{_{21}D} = i v_{i - 1}, v_i^{_{31}D} = v_i^{_{32}D} = 0 $
\newline

$ w_i^{_{12}D} = (\lambda_1 - \lambda_2 - i)w_{i + 1}, $

$ w_i^{_{13}D} = (\lambda_1 - \lambda_2 - i)r_{i + 1}, $

$ w_i^{_{23}D} = (\lambda_2 + \lambda_3 + i + 1)r_i, $

$ w_i^{_{21}D} = -u_i - i w_{i - 1}, w_i^{_{31}D} = v_i, w_i^{_{32}D} = 0, $
\newline

$ u_i^{_{12}D} = - w_i + (\lambda_1 - \lambda_2 - i)u_{i + 1}, $

$ u_i^{_{13}D} = (i - \lambda_1 - \lambda_3 - 1)r_i, $

$ u_i^{_{23}D} = -i r_{i - 1}, $

$ u_i^{_{21}D} = i u_{i - 1}, u_i^{_{31}D} = 0, u_i^{_{32}D} = v_i, $
\newline

$ r_i^{_{12}D} = (\lambda_1 - \lambda_2 - i)r_{i + 1}, $

$ r_i^{_{13}D} = r_i^{_{23}D} = 0, $

$ r_i^{_{21}D} = i r_{i - 1}, r_i^{_{31}D} = -u_i, r_i^{_{32}D} = w_i. $


Отсюда следует, что $ v_0, \ldots, v_t \in L_{\lambda} $, а поэтому $ v_i^{_{13}D} $ и $ v_{i + 1}^{_{23}D} $ 
тоже принадлежат $ L_{\lambda} $ при $ 0 \leq i < t $. Для $ 0 \leq i < t $ представим $ v_i^{_{13}D} $ и $ v_{i + 1}^{_{23}D} $ 
как линейную комбинацию векторов $ w_i, u_{i + 1} $ подпространства с весом $ (\lambda_1 - i - 1, \lambda_2 + i \,|\, \lambda_3) $. 
Зависимость выражается матрицей 

$$
\begin{pmatrix}
\lambda_1 + \lambda_3 - i & \lambda_1 - \lambda_2 - i \\
i + 1 & \lambda_2 + \lambda_3 + i + 1 \\
\end{pmatrix}
.$$

Её определитель $ \mbox{det}\, \lambda = (\lambda_1 + \lambda_3 + 1) (\lambda_2 + \lambda_3) $.
\\

(a) $\lambda$ регулярный.

Так как $ \mbox{det}\, \lambda \not\equiv 0 ~(\mbox{mod}\,p) $, то $ w_i, u_{i + 1} \in L_{\lambda} $, 
а следовательно и $ r_i \in L_{\lambda} $ для $ 0 \leq i < t $.
Получаем, что $ v_0, \ldots, v_t, w_0, \ldots, w_t, u_0, \ldots, u_t, r_0, \ldots, r_t $ составляют базис $ L_{\lambda} $. Следовательно,
$$ \chi(\lambda) = (x_1 x_2)^{\lambda_2 - 1} x_3^{\lambda_3} p_t (x_1, x_2) [\,x_1 x_2 + x_1 x_3 + x_2 x_3 + x_3^2\,] $$
и $ \mbox{dim}\,(L_{\lambda}) = 4 (t + 1). $

(b) $\lambda$ критический.

Так как $ \lambda_2 + \lambda_3 \not\equiv 0 ~(\mbox{mod}\,p) $ и $ v_0^{_{23}D} = (\lambda_2 + \lambda_3) u_0 $, то $ u_0 \in L_{\lambda}. $ 
Кроме того, $ \lambda_1 + \lambda_3 - t \equiv \lambda_2 + \lambda_3 \not\equiv 0 ~(\mbox{mod}\,p) $ и $ v_t^{_{13}D} = 
(\lambda_1 + \lambda_3 - t) w_t $, поэтому $ w_t \in L_{\lambda} $.

$ v_i^{_{13}D} $ и $ v_{i + 1}^{_{23}D} $ линейно зависимы, поэтому рассмотрим только 
$ q_i = v_i^{_{13}D} = -(i + 1)w_i + (t - i)u_{i + 1} \in L_{\lambda} $ при $ 0 \leq i < t-1 $.

Выясним, какие векторы порождаются векторами $u_0, w_t $ и $ q_i $: \\
$ w_t^{_{12}D} = (\lambda_1 - \lambda_2 - t)w_i = 0, w_t^{_{13}D} = 0, w_t^{_{23}D} = (\lambda_2 + \lambda_3 + t + 1)r_i = (\lambda_1 + \lambda_3 + 1)r_i = 0, \\
u_0^{_{12}D} = -w_0 + (\lambda_1 - \lambda_2)u_1 = q_0, u_0^{_{13}D} = (-\lambda_1 - \lambda_3 - 1)r_i = 0, u_0^{_{23}D} = 0, \\
q_i^{_{12}D} = -(i + 1)(t - i)w_{i + 1} + (t - i)(-w_{i + 1} + (t - (i + 1))u_{i + 2} = (t - i)q_{i + 1}, \\
q_i^{_{13}D} = -(i + 1)(t - i)r_{i + 1} + (t - i)(i + 1 -\lambda_1 - \lambda_3 - 1)r_{i + 1} = (t - i)(-\lambda_1 - \lambda_3 - 1)r_{i + 1} = 0, \\
q_i^{_{23}D} = -(i + 1)(\lambda_2 + \lambda_3 + i + 1) - (t - i)(i + 1)r_i = -(i + 1)(t + \lambda_2 + \lambda_3 + 1)r_i = 0. $ 
Таким образом, новые векторы не появляются, следовательно, векторы $ v_0, \ldots, v_t, u_0, w_t, q_0, \ldots, q_{t - 1} $ 
составляют базис $ L_{\lambda} $. Учитывая, что вес $q_i$ cовпадает с весом $w_i$, получаем 

$$ \chi(\lambda) = (x_1 x_2)^{\lambda_2 - 1} x_3^{\lambda_3} [\,p_t (x_1, x_2) x_1 x_2 + p_{t + 1} (x_1, x_2) x_3\,] $$
и $ \mbox{dim}\,(L_{\lambda}) = 2t + 3 $.

(c) $\lambda$ сильно критический.

Аналогично предыдущему пункту рассматриваем только $ q_i = v_i^{_{13}D} = 
(\lambda_1 + \lambda_3 - i)w_i + (t - i)u_{i + 1} = (t - i)(w_i + u_{i + 1}) \in L_{\lambda} $ при $ 0 \leq i < t $. 

$ q_i^{_{12}D} = (t - i)q_{i + 1}, q_i^{_{13}D} = 0, q_i^{_{23}D} = 0 $ при $ 0 \leq i < t $. Кроме того, $ v_0^{_{23}D} = 
(\lambda_2 + \lambda_3)u_0 $ и $ v_t^{_{13}D} = (\lambda_1 + \lambda_3 - t)w_t = 0 $, поэтому $ u_0, w_t \notin L_{\lambda} $. Следовательно, 
$$ \chi(\lambda) = (x_1 x_2)^{\lambda_2} x_3^{\lambda_3} [\,p_t (x_1, x_2) + p_{t - 1} (x_1, x_2) x_3 \,] $$
и $ \mbox{dim}\,(L_{\lambda}) = 2t + 1 $.

\end{proof}

\begin{remark}
Если $ \lambda_2 = 0 $, то $ \lambda_3 '' = 0 $, поэтому вес $\lambda$ является сильно критическим, следовательно
$$ \chi(\lambda) = x_3^{\lambda_3} [\,p_t (x_1, x_2) + p_{t - 1} (x_1, x_2) x_3 \,] 
\prod\limits_{i = 0}^{s}p_{k_i} ~(x_1^{p^{i + 1}}, x_2^{p^{i + 1}})
$$
\end{remark}
