Пусть $ \lambda = (\lambda_1, \lambda_2 \,|\, \lambda_3) $ -- полиномиальный вес и $L_{\lambda}$ - 
неприводимый подмодуль костандартного модуля $V = \nabla(\lambda)$ со старшим весом $\lambda$. 
Согласно \cite{brundan_kujawa}, если вес $\lambda$ является старшим весом $ L_{\lambda} $, 
то либо $ \lambda_2 > 0 $, либо $ \lambda_2 = 0 $ и $ p \,|\, \lambda_3 $.

\begin{definition}
Назовем вес $ \lambda = (\lambda_1, \lambda_2 \,|\, \lambda_3) $

-- регулярным, если $ (\lambda_1 + \lambda_3 + 1)(\lambda_2 + \lambda_3) \not\equiv 0 ~(\mbox{\textup{mod}}\,p); $

-- критическим, если $ \lambda_1 + \lambda_3 + 1 \equiv 0, \mbox{ но } \lambda_2 + \lambda_3 \not\equiv 0 ~(\mbox{\textup{mod}}\,p); $

-- сильно критическим, если $ \lambda_2 + \lambda_3 \equiv 0 ~(\mbox{\textup{mod}}\,p). $
\end{definition}

\begin{remark} Если $\lambda$ -- критический, то $ t \neq p - 1 $, т.к. в противном случае \\
$ \lambda_2 + \lambda_3 \equiv 0 ~(\mbox{\textup{mod}\, p}) $. 
\end{remark}

Обозначим $d = c_{11} c_{22} - c_{12} c_{21}, y_1 = \frac{c_{22} c_{13} - c_{12} c_{23}}{d}, y_2 = \frac{-c_{21} c_{13} + c_{11} c_{23}}{d}$

Определим следующие элементы:

$$ v_i = d^{\lambda_2} c_{11}^{\lambda_1 - \lambda_2 - i} c_{12}^i (c_{33}^{\lambda_3} - 
\lambda_3 c_{33}^{\lambda_3 - 1} (c_{31} y_1 + c_{32} y_2) + \lambda_3 (\lambda_3 - 1) c_{33}^{\lambda_3 - 2} c_{31} y_1 c_{32} y_2) $$ 
веса $(\lambda_1 - i, \lambda_2 + i\mid\lambda_3)$,

$$ w_i = d^{\lambda_2} c_{11}^{\lambda_1 - \lambda_2 - i} c_{12}^i (c_{33}^{\lambda_3} - \lambda_3 c_{33}^{\lambda_3 - 1} c_{32} y_2) y_1 $$ 
веса $(\lambda_1 - i - 1, \lambda_2 + i\mid\lambda_3 + 1)$,

$$ u_i = d^{\lambda_2} c_{11}^{\lambda_1 - \lambda_2 - i} c_{12}^i (c_{33}^{\lambda_3} - \lambda_3 c_{33}^{\lambda_3 - 1} c_{31} y_1) y_2 $$ 
веса $(\lambda_1 - i, \lambda_2 + i - 1\mid\lambda_3 + 1)$,

$$ r_i = d^{\lambda_2} c_{11}^{\lambda_1 - \lambda_2 - i} c_{12}^i c_{33}^{\lambda_3} y_1 y_2 $$ 
веса $(\lambda_1 - i - 1, \lambda_2 + i - 1\mid\lambda_3 + 2)$. 
Они порождают $H^0 (\lambda)$ как суперпространство для любого (не обязательно полиномиального) старшего веса $\lambda$. 
Обоснование можно найти в \cite{some_properties_supergroups}.

Суперпроизводные $_{ij}D$ определяются следующим действием на элементах 
$A(2 \,|\, 1)$: $ (c_{kl}) {_{ij}D} = \delta_{li} c_{kl} $, где $\delta_{li}$ -- символ Кронекера.

Обозначим через $\chi(\lambda)$ формальный характер простого модуля $L_{\lambda}$ 
и через $p_j (x_1, x_2) = \sum\limits_{0 \leq i \leq j} x_1^{i} x_2^{j - i}$ полную симметрическую функцию от $x_1, x_2$ степени $j$.
